\documentclass[a4paper,12pt]{article}

\usepackage[polish]{babel}
\usepackage[utf8]{inputenc}

\usepackage{blindtext}

\usepackage{amsmath}
\usepackage{amssymb}
\usepackage{graphicx}
\usepackage[margin=0.5in]{geometry}

%\textwidth=8in \textheight=9in 
%\oddsidemargin=0.5in
%\topmargin=-1in

\author{Jan Kowalski}
\title{Bez tytułu}

\frenchspacing

\DeclareMathOperator{\Q}{\mathbb{Q}}
\newcommand{\R}{\mathbb{R}}
\newcommand{\N}{\mathbb{N}}

\newtheorem{tw}{Twierdzenie}
\newtheorem{lem}{Lemat}
\newtheorem{wn}{Wniosek}
\newtheorem{fakt}{Fakt}
\newtheorem{uw}{Uwaga}
\newtheorem{defin}{Definicja}
\newtheorem{prz}{Przykład}
\newenvironment{dow}{\par \noindent \emph{Dowód.\ }}{\hfill $\Box$ \par}

\begin{document}
	\maketitle
	\newpage
	\tableofcontents
	\newpage
	
	\section{Pochodne}
	Lagrange
		
	\section{Całki}
	\begin{tw}
		\label{O trzech ciągach}
		Jeżeli mamy ciągi $(a_n)$, $(b_n)$, $(c_n)$ takie, że
		$$ \forall_{n \in \N} \exists_{q \in \Q}  a_n \leqslant b_n < c_n \Rightarrow \lim_{n \rightarrow \infty} b_n \Leftrightarrow $$
	\end{tw}
	\begin{dow}
		To jest oczywiste. $1\frac{4}{5} \cdot 2137$
	\end{dow}
	
	\begin{cases}
		$2x + y = 0$ & $x \geq 0$\\
		$3x + y = -2$ & wpp \\
	\end{cases}
	
	$$ \frac{x^4 + 1}{e^{2x} - \pi} < \varepsilon $$
	
	\subsection{Całki zwykłe}
	\blindtext	
	
	\subsection{Całki niezwykłe}
	\blindtext
	
	\section{Algebra liniowa}
	Cronecker  
	
\end{document}