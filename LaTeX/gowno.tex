\documentclass[12pt]{report}

\usepackage{amsmath}
\usepackage{amssymb}
\usepackage{graphicx}

\usepackage[polish]{babel}
\usepackage[utf8]{inputenc}
\usepackage{t1enc}


\textwidth=16cm \textheight=23cm \oddsidemargin=0.5cm
\topmargin=-1cm

\DeclareMathOperator{\Q}{\mathbb{Q}}
\def\R{\mathbb{R}} 

\newtheorem{tw}{Twierdzenie}
\newtheorem{lem}{Lemat}
\newtheorem{wn}{Wniosek}
\newtheorem{fakt}{Fakt}
\newtheorem{uw}{Uwaga}
\newtheorem{defin}{Definicja}
\newtheorem{prz}{Przykład}
\newenvironment{dow}{\par \noindent \emph{Dowód.\ }}{\hfill $\Box$ \par}

\begin{document}
	\thispagestyle{empty}
	\begin{tw} 
		\label{bolwei}
		(Bolzano-Weierstrassa) \newline
		Każdy ograniczony ciąg posiada \emph{podciag zbieżny}.
	\end{tw}
	\begin{dow}
		Ależ to jest oczywiste. XDDD
		$$\lim_{n\to\infty}f_n(x) = 21.37.$$
	\end{dow}
	\begin{tw} 
		\label{abel}
		(Abela) \newline
		Jest niewarte uwagi.
	\end{tw}
	\begin{dow}
		Tak jak w (\ref{bolwei}).
		$$\lim_{n\to\infty}f_n(x) = 14.88.$$
	\end{dow}
	\begin{figure}[h]
		\begin{center}
			\includegraphics[width=2in, height=1in]{rys1.jpg}
		\end{center}
		\caption{To jest okrąg.}
	\end{figure}
\end{document}