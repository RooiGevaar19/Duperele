\documentclass[a4paper,12pt]{article}

% ========== PACZKI
% podstawka
\usepackage{polski}
\usepackage[utf8]{inputenc}

\usepackage{amssymb}
\usepackage{verse}
\usepackage{epigraph}
\usepackage{float}
\usepackage{enumitem}
\usepackage[margin=0.25in]{geometry} % margines strony 
\usepackage{titling} % do użytku dowolnego tytułów

% format daty
\usepackage{datetime}
\newdateformat{ymd}{\THEYEAR-\twodigit{\THEMONTH}-\twodigit{\THEDAY}}
\newdateformat{mdy}{\twodigit{\THEMONTH}/\twodigit{\THEDAY}/\THEYEAR}
\newdateformat{dmy}{\twodigit{\THEDAY}.\twodigit{\THEMONTH}.\THEYEAR}


% liczniki do tabel
\usepackage{tabularx}
\usepackage{array,etoolbox}

\preto\tabular{\setcounter{magicrownumbersA}{0}}
\newcounter{magicrownumbersA}
\newcommand\rownumberA{\stepcounter{magicrownumbersA}\arabic{magicrownumbersA}}

\preto\tabular{\setcounter{magicrownumbersB}{0}}
\newcounter{magicrownumbersB}
\newcommand\rownumberB{\stepcounter{magicrownumbersB}\arabic{magicrownumbersB}}

% kolorki 
\usepackage{xcolor}
\usepackage{colortbl}

\definecolor{LightCyan}{rgb}{0.88,1,1}

% hiperłącza
\usepackage{hyperref}

\hypersetup{
    colorlinks=true,
    linkcolor=blue,
    filecolor=magenta,      
    %urlcolor=cyan,
    urlcolor=black,
}

% ============ Funkcje

\frenchspacing

\newcommand{\tytul}[1]{%
	\begin{center}%
	\LARGE\textbf{\MakeUppercase{#1}}%
	\end{center}%
}

\newcommand{\mail}[1]{%
    \href{mailto:me@#1}{#1}%
}

\newcommand{\pozycjaA}[8]{%
    \hline %
    #1 & #2 – #3 & #4 & #5 & #6 & #7 & #8 \\ %
}

\newcommand{\pozycjaB}[7]{%
    \hline %
    #1 & #2 – #3 & #4 & #5 & #6 & #7 \\ %
}

% ============ AUTORZY

\author{Robert Johansson}
\title{Dziennik praktyk}
\date{\today}

% ============ DOKUMENT

\begin{document}
    \begin{center}
    	\tytul{\thetitle}
    \end{center}

    \begin{tabular}{l m{2in}}
        \textbf{Data:}  & \dmy\today \\
        \textbf{Autor:} & \theauthor \\
    \end{tabular}

    \section{Informacje o praktykancie:}
    \begin{tabular}{l p{5in}}
        \textbf{Imię i nazwisko:} & Robert Johansson \\ 
        \textbf{Numer indeksu:}   & 232137 \\ 
        \textbf{Uczelnia:}        & Uniwersytet Gdański \\ 
        \textbf{Wydział:}         & Wydział Matematyki, Fizyki i Informatyki \\ 
        \textbf{Kierunek:}        & Matematyka \\ 
        \textbf{Specjalność:}     & nauczycielska \\
        \textbf{Stopień:}         & studia licencjackie \\ 
        \textbf{Rok studiów:}     & trzeci \\  
        \textbf{Adres e-mail:}    & \mail{wasacz252@gmail.com} \\ 
    \end{tabular}

    \section{Informacje o szkole:}
    \begin{tabular}{l p{5in}}
        \textbf{Nazwa:}          & Szkoła Podstawowa 2137 w Wadowicach im. Jana Pawła 2 \\ 
        \textbf{Rodzaj:}         & szkoła podstawowa \\ 
        \textbf{Adres pocztowy:} & ul. Kremówki 21/37 42-137 Wadowice \\
        \textbf{Adres e-mail:}   & \mail{sekretariat@spwadowice2137.pl} \\
    \end{tabular}

    \section{Informacje o opiekunie praktyk:}
    \begin{tabular}{l m{2in}}
        \textbf{Imię i nazwisko:} & Karol Wojtyła \\ 
        \textbf{Adres e-mail:}    & \mail{pawulon2137@gmail.com} \\ 
    \end{tabular}

    \newpage
    \section{Wykaz lekcji hospitowanych}

    \tiny
    \begin{tabularx}{\textwidth}{|@{\makebox[0.2in][r]{\rownumberA\space}} | c | c | c | c | c | X | X |}
        \hline
            \rowcolor{LightCyan}
            \multicolumn{1}{|@{{\makebox[0.2in][c]{\colorbox{LightCyan}{Lp}}}} | c |}{Data} & Godziny zajęć & Klasa & Tryb zajęć & Nauczyciel prowadzący & Temat & Uwagi \\  
            \pozycjaA{15.10.2020}{8:00}{8:45}{8c}{stacjonarne}{Petronella Iksińska}{Przekształcanie wzorów i wyznaczanie wartości}{}
            \pozycjaA{15.10.2020}{8:50}{9:35}{8b}{stacjonarne}{Petronella Iksińska}{Równania, c.d.}{}
            \pozycjaA{16.10.2020}{8:00}{8:45}{8c}{stacjonarne}{Petronella Iksińska}{Omówienie sprawdzianu}{}
        \hline
    \end{tabularx}

    ~

    ~

    \section{Wykaz lekcji prowadzonych}

    \tiny
    \begin{tabularx}{\textwidth}{|@{\makebox[0.2in][r]{\rownumberB\space}} | c | c | c | c | X | X |}
        \hline
            \rowcolor{LightCyan}
            \multicolumn{1}{|@{\makebox[0.2in][c]{\colorbox{LightCyan}{Lp}}} | c |}{Data} & Godziny zajęć & Klasa & Tryb zajęć & Temat & Uwagi \\
            \pozycjaB{22.10.2020}{8:00}{8:45}{6a}{stacjonarne}{Kąty}{}
            \pozycjaB{22.10.2020}{8:50}{9:35}{6a}{stacjonarne}{Kąty c.d.}{}
            \pozycjaB{23.10.2020}{8:00}{8:45}{6a}{stacjonarne}{Kąty w trójkątach i czworokątach}{}
            \pozycjaB{23.10.2020}{8:50}{9:35}{6b}{stacjonarne}{Pole równoległoboku i rombu}{}
            \pozycjaB{23.10.2020}{9:45}{10:30}{6c}{stacjonarne}{Kąty w trójkątach i czworokątach}{}
        \hline
    \end{tabularx}

\end{document}