\documentclass[letterpaper,12pt]{article}

\usepackage[polish]{babel}
\usepackage[utf8]{inputenc}
\usepackage{blindtext}
\usepackage{titling}
\usepackage{setspace}

\usepackage{amsmath}
\usepackage{amssymb}
\usepackage{graphicx}
\usepackage{enumitem}
\usepackage[margin=0.5in]{geometry}

%\textwidth=8in \textheight=9in 
%\oddsidemargin=0.5in
%\topmargin=-1in

\author{Paweł Lipkowski}
\title{Untitled}
\date{\today} % use \date{\today} or comment this line to have a current date

%\frenchspacing

\DeclareMathOperator{\Q}{\mathbb{Q}}
\newcommand{\R}{\mathbb{R}}
\newcommand{\N}{\mathbb{N}}

\newtheorem{tw}{Twierdzenie}
\newtheorem{lem}{Lemat}
\newtheorem{wn}{Wniosek}
\newtheorem{fakt}{Fakt}
\newtheorem{uw}{Uwaga}
\newtheorem{defin}{Definicja}
\newtheorem{prz}{Przykład}
\newenvironment{dow}{\par \noindent \emph{Dowód.\ }}{\hfill $\Box$ \par}

\begin{document}
    %\maketitle       % uncomment it to include the title page
    %\tableofcontents % uncomment it to include the table of contents
    %\blindtext        % comment it to remove Lorem Ipsum
    % --------------------------------------------------------------
    % write document down below
       	
	\begin{tabular}{l p{2.5in}}
		\textbf{Data:}  & \thedate \\
		\textbf{Autor:} & \theauthor \\
	\end{tabular}
   	\setstretch{1.25}
   	
    \section{Zadanie 5}
    Oblicz:
    \begin{enumerate}[label=\alph*)]
        \item $ 1\frac{1}{7} + \frac{3}{5} = %
       		\frac{8}{7} + \frac{3}{5} = %
        	\frac{40}{35} + \frac{21}{35} = %
        	\frac{61}{35} = 1\frac{26}{35} $
        \item $2\frac{3}{4} + 1\frac{2}{3} = %
        	\frac{11}{4} + \frac{5}{3} = %
        	\frac{33}{12} + \frac{20}{12} = %
        	\frac{53}{12} = 4\frac{5}{12}$
        \item $4\frac{3}{4} - 1\frac{2}{5} =%
        	4\frac{15}{20} - 1\frac{8}{20} =%
        	4\frac{15}{20} - 1 - \frac{8}{20} =%
        	3\frac{15}{20} - \frac{8}{20} =%
        	3\frac{7}{20} $
        \item $5\frac{1}{4} - 3\frac{1}{2} = %
        	5\frac{1}{4} - 3\frac{2}{4} = %
        	5\frac{1}{4} - 3 - \frac{2}{4} = %
        	2\frac{1}{4} - \frac{2}{4} = %
        	\frac{9}{4} - \frac{2}{4} = \frac{7}{4} = 1\frac{3}{4}$
    \end{enumerate}
    %\newpage
    \section{Zadanie 6}
    Oblicz:
    \begin{enumerate}[label=\alph*)]
    	\item $18 * 4\frac{2}{3} = % 
    		18 * \frac{14}{3} = %
    		6 * 14 = 84$
    	\item $1\frac{1}{2} * 2\frac{2}{3} = %
    		\frac{3}{2} * \frac{8}{3} = %
    		\frac{8}{2} = 4 $
    	\item $ 2\frac{1}{7} : 12 = % 
    		\frac{15}{7} : 12 = %
    		\frac{15}{7} * \frac{1}{12} = %
    		\frac{5}{7} * \frac{1}{4} = \frac{5}{28} $
    	\item $ 7\frac{1}{3} : 1\frac{2}{9} = %
    		\frac{22}{3} : \frac{11}{9} = %
    		\frac{22}{3} * \frac{9}{11} = %
    		\frac{2}{3} * \frac{9}{1} = %
    		\frac{2}{1} * \frac{3}{1} = %
    		2 * 3 = 6$
    \end{enumerate}
    %\newpage
    \section{Zadanie 7}
    Oblicz:
    \begin{enumerate}[label=\alph*)]
    	\item $ (\frac{1}{10} + \frac{8}{15}) * 30 = \\=%
    		30*\frac{1}{10} + 30*\frac{8}{15} = \\=%
    		\frac{30}{1}*\frac{1}{10} + \frac{30}{1}*\frac{8}{15} = \\=%
    		3 + 2*8 =\\= 3 + 16 = 19 $
    	\item $ 5\frac{2}{3} - 4*1\frac{1}{8} =\\=%
    		5\frac{2}{3} - 4*\frac{9}{8} =\\=%
    		5\frac{2}{3} - \frac{36}{8} =\\=%
    		5\frac{2}{3} - 4\frac{4}{8} =\\=%
    		5\frac{2}{3} - 4\frac{1}{2} =\\=%
    		5\frac{4}{6} - 4\frac{3}{6} =\\=%
    		\frac{34}{6} - \frac{27}{6} = \frac{7}{6} = 1\frac{1}{6} $
    \end{enumerate}
    %\newpage
    \section{Zadanie 8}
    Oblicz:
    \begin{enumerate}[label=\alph*)]
    	\item $\frac{5}{8} * 7\frac{1}{5} =\\=
    		\frac{5}{8} * \frac{36}{5} =\\=
    		\frac{36}{8} = 4\frac{4}{8} = 4\frac{1}{2} \neq 5\frac{23}{40} $
    		
    		\textbf{Odp.} FAŁSZ
    	\item $\frac{3}{7} * \frac{7}{15} =\\= 
    		\frac{3}{1} * \frac{1}{15} = \frac{3}{15} = \frac{1}{5} $
    		
    		\textbf{Odp.} PRAWDA
    \end{enumerate}
\end{document}