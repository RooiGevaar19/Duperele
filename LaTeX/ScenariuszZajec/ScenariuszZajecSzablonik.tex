\documentclass[a5paper,10pt,bibliography=totoc]{article}
\usepackage{polski}
\usepackage[utf8x]{inputenc}
\usepackage{amsmath}
\usepackage{titlepic}
\usepackage{graphicx}
\usepackage{wallpaper}
\usepackage{eso-pic}
\usepackage{amssymb}
\usepackage{verse}
\usepackage{epigraph}
\usepackage{float}
\usepackage{xcolor}
\usepackage{enumitem}
\usepackage[margin=0.75in]{geometry}


\frenchspacing

\DeclareMathOperator{\Q}{\mathbb{Q}}
\newcommand{\R}{\mathbb{R}}
\newcommand{\N}{\mathbb{N}}

\newtheorem{tw}{Twierdzenie}
\newtheorem{lem}{Lemat}
\newtheorem{wn}{Wniosek}
\newtheorem{fakt}{Fakt}
\newtheorem{uw}{Uwaga}
\newtheorem{defin}{Definicja}
\newtheorem{prz}{Przykład}
\newenvironment{dow}{\par \noindent \emph{Dowód.\ }}{\hfill $\Box$ \par}

\newcommand\tab[1][0.5in]{\hspace*{#1}}

\newcommand{\tytul}[1]{%
	\begin{center}%
	\LARGE\textbf{#1}%
	\end{center}%
}

\newcommand{\podpunkt}[2]{\textbf{#1} #2}

\newcommand{\zadanie}[4]{%
	#1 \textit{(ok. #3min)}%
	\begin{figure}[h!]%
		\centering%
  		\includegraphics[width=#4]{#2}%
  		%\caption{#1}%
  		%\label{fig:boat1}%
	\end{figure}%
}

\begin{document}
\begin{center}
	\tytul{SCENARIUSZ ZAJĘĆ}
\end{center}


\begin{tabular}{l m{7cm}}
\textbf{Data:}               & 22.10.2020 \\
\textbf{Autor:}              & Paweł Lipkowski \\
\textbf{Klasa:}              & VIa, klasa Szkoły Podstawowej \\ 
\textbf{Temat lekcji:}       & Kąty – wprowadzenie do tematu \\
\textbf{Cel lekcji:}         & Uczeń wskazuje w~dowolnym kącie ramiona i~wierzchołek; rozpoznaje kąt prosty, ostry i~rozwarty;
porównuje kąty \\
\textbf{Czas lekcji:}        & 45 minut \\
\textbf{Metody:}             & Pogadanka, objaśnianie, zadania praktyczne \\
\textbf{Literatura:}         & podręcznik Matematyka z~Plusem~6, GWO +~ćwiczenia \\
\textbf{Środki dydaktyczne:} & multipodręcznik Matematyka z~Plusem~6, GWO; kartka A4; nożyczki; ekierka; może być program obróbki grafiki, np.~GIMP lub Paint
\end{tabular}


\section{Plan ogólny}

\begin{enumerate}
    \item Przywitanie
    \item Wprowadzenie do tematu
    \item Rozwiązywanie zadań
    \item Podsumowanie lekcji
\end{enumerate}


\section{Plan szczegółowy:}

\begin{enumerate} %[label=\arabic*)] %\nopagebreak[4]
	\item Przywitanie \textit{(sumarycznie 4min)}
	
Nauczyciel wchodzi do sali, a następnie wita się z uczniami i się przedstawia. Potem zostaje sprawdzona obecność. Po sprawdzeniu obecności ogłasza nowy temat i prosi o otwarcie zeszytów. 

	\item Wprowadzenie do tematu \textit{(sumarycznie 13min)}
	\begin{enumerate}[label=\alph*)]
		\item Luźna pogadanka o kątach w codziennym życiu. \textit{(2min)}
		\item Wprowadzenie pojęcia wierzchołka i ramion kąta. Wykonanie ćwiczeń A,B/124. \textit{(5min)}
		\item Wprowadzenie pojęć kąta ostrego, prostego, rozwartego, półpełnego, wklęsłego i pełnego. Wykonanie ćwiczeń C,D/126. \textit{6min)}
	\end{enumerate}
	%\newpage
	
    \item Rozwiązywanie zadań \textit{(maksymalnie ok. 23-27min)} 
	\begin{enumerate}[label=\alph*)]
		\item \zadanie{Zadania 1,2/127}{1,2-127.png}{7}{3in}
		
		Uczniowie wykonują to zadanie w zeszycie. Nauczyciel wyrywkowo sprawdza dzieła uczniów i po upływie paru minut przedstawia na tablicy przykłady rozwiązania tych zadań.
		\newline

		\item \zadanie{Zadanie 3/127}{3-127.png}{5}{3in}
		
		Zadanie zostaje wykonane ustnie.
		\newline
		
		\item \zadanie{Zadanie 4/127}{4-127.png}{6}{2in}
		
		Uczniowie wykonują to zadanie w zeszycie. Nauczyciel wyrywkowo sprawdza dzieła uczniów i po upływie paru minut przedstawia na tablicy przykłady rozwiązania tych zadań.
		\newline\newpage
		
		\item \zadanie{Zadanie 5/127}{5-127.png}{2}{2in} \nopagebreak[4]
		
		Zadanie zostaje wykonane ustnie.
		\newline
		
		\item \zadanie{Zadanie 6/127}{6-127.png}{3}{3in} \nopagebreak[4]
		
		Zadanie zostaje wykonane ustnie, jednakże nauczyciel poleca aktywność w postaci narysowania przykładów albo na tablicy.
		\newline
		
		\item \zadanie{Zadanie 7/127}{7-127.png}{4}{3in} \nopagebreak[4]
		
		\textbf{Zadanie dodatkowe}, gdyby wcześniejsze zadania przebiegły przed czasem. Zadanie zostaje wykonane ustnie, jednakże nauczyciel poleca aktywność w postaci narysowania przykładów albo na tablicy.
		\newline

	\end{enumerate}
		
    \item Podsumowanie lekcji \textit{(sumarycznie 3min)}

Zapytanie się o odczucia uczniów odnośnie ilości zdobytych wiadomości, zrozumienia tematu, zadanie zadań domowych (\textbf{ćw. 1-3/56}) oraz pożegnanie się z uczniami. \textit{(3min)}
\end{enumerate}

\end{document}